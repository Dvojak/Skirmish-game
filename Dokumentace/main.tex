% ŠABLONA PRO PSANÍ ZÁVĚREČNÉ STUDIJNÍ PRÁCE
%%%%%%%%%%%%%%%%%%%%%%%%%%%%%%%%%%%%%%%%%%%%
% Autor: Jakub Dokulil (kubadokulil99@gmail.com)
% Tato šablona byla vytvořena tak, aby pomocí ní mohli v systému LaTeX soutěžící sázet své práce a zároveň odpovídala požadavkům na formátování vyplývajícím z wordové šablony umístěné na webu soc.cz.
%
\documentclass[12pt, a4paper,
%oneside,      %% -- odkomentujte, pokud chcete svou práci mít pouze jednostrannou, mezera pro hřbet pak automaticky bude pouze na levé straně
twoside,        %% -- pro oboustranné práce, mezera pro hřbet následně střídá strany.
openright
]{report}



%% Nutné balíčky a nastavení
%%%%%%%%%%%%%%%%%%%%%%%%%%%%

%% Proměnné
\newcommand\obor{INFORMAČNÍ TECHNOLOGIE} %% -- napiš číslo a název tvého oboru
\newcommand\kodOboru{18-20-M/01} %% -- napiš číslo a název tvého oboru
\newcommand\zamereni{se zaměřením na počítačové sítě a programování} %% -- napiš číslo a název tvého oboru
\newcommand\skola{Střední škola průmyslová a umělecká, Opava} %% vyplň název školy
\newcommand\trida{IT4} %% vyplň jméno svého konzultanta
\newcommand\jmenoAutora{Walter Oboňa}  %% vyplň své jméno
\newcommand\skolniRok{2025/26} %% vyplň rok
\newcommand\datumOdevzdani{6. 1. 2026} %% vyplň rok
\newcommand\nazevPrace{Skirmish game} %% vyplň název své práce

\title{\nazevPrace} %% -- Název tvé práce
\author{\jmenoAutora} %% -- tvé jméno
\date{\datumOdevzdani} %% -- rok, kdy píšeš SOČku

\usepackage[top=2.5cm, bottom=2.5cm, left=3.5cm, right=1.5cm]{geometry} %% nastaví okraje, left -- vnitřní okraj, right -- vnější okraj

\usepackage[czech]{babel} %% balík babel pro sazbu v češtině
\usepackage[utf8]{inputenc} %% balíky pro kódování textu
\usepackage[T1]{fontenc}
\usepackage{cmap} %% balíček zajišťující, že vytvořené PDF bude prohledávatelné a kopírovatelné

\usepackage{graphicx} %% balík pro vkládání obrázků

\usepackage{subcaption} %% balíček pro vkládání podobrázků

\usepackage{hyperref} %% balíček, který v PDF vytváří odkazy

\linespread{1.25} %% řádkování
\setlength{\parskip}{0.5em} %% odsazení mezi odstavci


\usepackage[pagestyles]{titlesec} %% balíček pro úpravu stylu kapitol a sekcí
\titleformat{\chapter}[block]{\scshape\bfseries\LARGE}{\thechapter}{10pt}{\vspace{0pt}}[\vspace{-22pt}]
\titleformat{\section}[block]{\scshape\bfseries\Large}{\thesection}{10pt}{\vspace{0pt}}
\titleformat{\subsection}[block]{\bfseries\large}{\thesubsection}{10pt}{\vspace{0pt}}


\usepackage{tocloft} % Balíček umožní přizpůsobit vzhled tabulky obsahu
\setlength{\cftbeforechapskip}{0pt}  % Menší rozestup pro kapitoly
\setlength{\cftbeforesecskip}{0pt}   % Menší rozestup pro sekce

\setcounter{secnumdepth}{2}
\setcounter{tocdepth}{1}
\usepackage{fancyhdr}
\pagestyle{fancy}
\renewcommand{\headrulewidth}{0.025pt}

\usepackage{booktabs}

\usepackage{url}

%% Balíčky co se můžou hodit :) 
%%%%%%%%%%%%%%%%%%%%%%%%%%%%%%%

\usepackage{pdfpages} %% Balíček umožňující vkládat stránky z PDF souborů, 

\usepackage{upgreek} %% Balíček pro sazbu stojatých řeckých písmen, třeba u jednotky mikrometr. Například stojaté mí: \upmu, stojaté pí: \uppi

\usepackage{amsmath}    %% Balíčky amsmath a amsfonts 
\usepackage{amsfonts}   %% pro sazbu matematických symbolů
\usepackage{esint}     %% pro sazbu různých integrálů (např \oiint)
\usepackage{mathrsfs}
\usepackage{helvet} % Helvet font
\usepackage{mathptmx} % Times New Roman
\makeatletter
\@namedef{ver@figureversions.sty}{9999/99/99}
\newcommand{\DeclareFigureVersion}[2]{}
\newcommand{\figureversion}[1]{}
\makeatother


\makeatletter
\providecommand{\superiorSup}{}
\providecommand{\textOsF}{}
\providecommand{\textTOsF}{}
\providecommand{\liningLF}{}
\providecommand{\liningTLF}{}
\providecommand{\tabularTab}{}
\providecommand{\proportionalProp}{}
\makeatother
\makeatletter
\providecommand{\superiorSup}{}
\providecommand{\textOsF}{}
\providecommand{\textTOsF}{}
\providecommand{\liningLF}{}
\providecommand{\liningTLF}{}
\providecommand{\tabularTab}{}
\providecommand{\proportionalProp}{}
\providecommand{\tabularmath}{}
\providecommand{\proportionalmath}{}
\makeatother

\usepackage{Oswald} % Oswald font


%% makra pro sazbu matematiky
\newcommand{\dif}{\mathrm{d}} %% makro pro sazbu diferenciálu, místo toho
%% abych musel psát '\mathrm{d}' mi stačí napsat '\dif' což je mnohem 
%% kratší a mohu si tak usnadnit práci

\usepackage{listings}
\usepackage{xcolor}

\renewcommand{\lstlistingname}{Kód}% Listing -> Algorithm
\renewcommand{\lstlistlistingname}{Seznam programových kódů}% List of Listings -> List of Algorithms

%% Definice 
\lstdefinelanguage{JavaScript}{
	morekeywords=[1]{break, continue, delete, else, for, function, if, in,
		new, return, this, typeof, var, void, while, with},
	% Literals, primitive types, and reference types.
	morekeywords=[2]{false, null, true, boolean, number, undefined,
		Array, Boolean, Date, Math, Number, String, Object},
	% Built-ins.
	morekeywords=[3]{eval, parseInt, parseFloat, escape, unescape},
	sensitive,
	morecomment=[s]{/*}{*/},
	morecomment=[l]//,
	morecomment=[s]{/**}{*/}, % JavaDoc style comments
	morestring=[b]',
	morestring=[b]"
}[keywords, comments, strings]


\lstdefinelanguage[ECMAScript2015]{JavaScript}[]{JavaScript}{
	morekeywords=[1]{await, async, case, catch, class, const, default, do,
		enum, export, extends, finally, from, implements, import, instanceof,
		let, static, super, switch, throw, try},
	morestring=[b]` % Interpolation strings.
}

\lstalias[]{ES6}[ECMAScript2015]{JavaScript}

% Nastavení barev
% Requires package: color.
\definecolor{mediumgray}{rgb}{0.3, 0.4, 0.4}
\definecolor{mediumblue}{rgb}{0.0, 0.0, 0.8}
\definecolor{forestgreen}{rgb}{0.13, 0.55, 0.13}
\definecolor{darkviolet}{rgb}{0.58, 0.0, 0.83}
\definecolor{royalblue}{rgb}{0.25, 0.41, 0.88}
\definecolor{crimson}{rgb}{0.86, 0.8, 0.24}

% Nastavení pro Python
\lstdefinestyle{Python}{
	language=Python,
	backgroundcolor=\color{white},
	basicstyle=\ttfamily,
	breakatwhitespace=false,
	breaklines=false,
	captionpos=b,
	columns=fullflexible,
	commentstyle=\color{mediumgray}\upshape,
	emph={},
	emphstyle=\color{crimson},
	extendedchars=true,  % requires inputenc
	fontadjust=true,
	frame=single,
	identifierstyle=\color{black},
	keepspaces=true,
	keywordstyle=\color{mediumblue},
	keywordstyle={[2]\color{darkviolet}},
	keywordstyle={[3]\color{royalblue}},
	literate=%
	{á}{{\'a}}1 {č}{{\v{c}}}1 {ď}{{\v{d}}}1 {é}{{\'e}}1 {ě}{{\v{e}}}1
	{í}{{\'i}}1 {ň}{{\v{n}}}1 {ó}{{\'o}}1 {ř}{{\v{r}}}1 {š}{{\v{s}}}1
	{ť}{{\v{t}}}1 {ú}{{\'u}}1 {ů}{{\r{u}}}1 {ý}{{\'y}}1 {ž}{{\v{z}}}1,		
	numbers=left,
	numbersep=5pt,
	numberstyle=\tiny\color{black},
	rulecolor=\color{black},
	showlines=true,
	showspaces=false,
	showstringspaces=false,
	showtabs=false,
	stringstyle=\color{forestgreen},
	tabsize=2,
	title=\lstname,
	upquote=true  % requires textcomp	
}


\lstdefinestyle{JSES6Base}{
	backgroundcolor=\color{white},
	basicstyle=\ttfamily,
	breakatwhitespace=false,
	breaklines=false,
	captionpos=b,
	columns=fullflexible,
	commentstyle=\color{mediumgray}\upshape,
	emph={},
	emphstyle=\color{crimson},
	extendedchars=true,  % requires inputenc
	fontadjust=true,
	frame=single,
	identifierstyle=\color{black},
	keepspaces=true,
	keywordstyle=\color{mediumblue},
	keywordstyle={[2]\color{darkviolet}},
	keywordstyle={[3]\color{royalblue}},
 literate=%
{á}{{\'a}}1 {č}{{\v{c}}}1 {ď}{{\v{d}}}1 {é}{{\'e}}1 {ě}{{\v{e}}}1
{í}{{\'i}}1 {ň}{{\v{n}}}1 {ó}{{\'o}}1 {ř}{{\v{r}}}1 {š}{{\v{s}}}1
{ť}{{\v{t}}}1 {ú}{{\'u}}1 {ů}{{\r{u}}}1 {ý}{{\'y}}1 {ž}{{\v{z}}}1,		
	numbers=left,
	numbersep=5pt,
	numberstyle=\tiny\color{black},
	rulecolor=\color{black},
	showlines=true,
	showspaces=false,
	showstringspaces=false,
	showtabs=false,
	stringstyle=\color{forestgreen},
	tabsize=2,
	title=\lstname,
	upquote=true  % requires textcomp
}

\lstdefinestyle{JavaScript}{
	language=JavaScript,
	style=JSES6Base,
}
\lstdefinestyle{ES6}{
	language=ES6,
	style=JSES6Base
}

\setlength{\headheight}{15pt}

%% Bordel pro práci - můžeš smáznout :) 
%%%%%%%%%%%%%%%%%%%

\usepackage{lipsum} %% balíček který píše lipsum (nesmyslný text, který se používá pro kontrolu typografie)

\AtBeginDocument{\clearpage\pagestyle{empty}}

%% Začátek dokumentu
%%%%%%%%%%%%%%%%%%%%
\begin{document}
	
	\pagestyle{empty}
	\pagenumbering{Roman}
	
	\cleardoublepage

%% Titulní stránka s informacemi
%%%%%%%%%%%%%%%%%%%%%%%%%%%%%%%%%%%%%%%%
	
	{\fontfamily{phv}\selectfont
		%% Logo školy
		\begin{figure}[h]
			\centering
			\includegraphics[width=0.6\linewidth]{image/logo-skoly.png} 
		\end{figure}
		
		
		%% Hlavička práce a její název (viz proměnná \nazev prace)
		%% \sffamily %%% bezpatkové písmo - sans serif
		{\bfseries %%% písmo na stránce je tučně
			\begin{center}
				\vspace{0.025 \textheight}
				\LARGE{ZÁVĚREČNÁ STUDIJNÍ PRÁCE}\\
				\large{dokumentace}\\
				\vspace{0.075 \textheight}
				\LARGE {\nazevPrace}\\
			\end{center}  
		}%%%
		
		\begin{figure}[h]
			\centering
			\includegraphics[width=0.8\linewidth]{image/programovani-02.jpg} 
		\end{figure}
		
		\vfill
		%% Uprava - misto floating table pouzijeme pevne zarovnani na spodek stranky
		\begin{center}
			\begin{tabular}{ll}
				\textbf{Autor:} & \jmenoAutora\\ 
				\textbf{Obor:} & \kodOboru { } \obor\\
				\textbf{} & \zamereni\\
				\textbf{Třída:} & \trida\\
				\textbf{Školní rok:} & \skolniRok\\
			\end{tabular}
		\end{center}		
	}
	
\cleardoublepage %% Zalomení dvojstránky
	
%% Stránka obsahující poděkování a prohlášení
%%%%%%%%%%%%%%%%%%%%%%%%%%%%%%%%%%%%%%%%%%%%%%%%%%%%%%%%


%% Prohlášení - povinné
%%%%%%%%%%%%%%%%%%%%%%%%%%%%
	\noindent{\large{\bfseries{Prohlášení}\\}}  %% uprav si koncovky podle toho na jaký rod se cítíš, vypadá to pak lépe :) 
	\noindent{Prohlašuji, že jsem závěrečnou práci vypracoval samostatně a uvedl veškeré použité 
		informační zdroje.\\}
	\noindent{Souhlasím, aby tato studijní práce byla použita k výukovým a prezentačním účelům na Střední průmyslové a umělecké škole v Opavě, Praskova 399/8.}
	\vfill
	\noindent{V Opavě \datumOdevzdani\\}
	\noindent
	\begin{minipage}{\linewidth}
		\hspace{9.5cm} 
		\begin{tabular}{@{}p{6cm}@{}}
			\dotfill \\
			Podpis autora
		\end{tabular}
	\end{minipage}
	
	\cleardoublepage %% Zalomení dvojstránky

%% Stránka obsahující abstrakt (anotaci)
%%%%%%%%%%%%%%%%%%%%%%%%%%%%%%%%%%%%%%%%%%%%%%%%%%%%%%%%	

%% Abstrakt v češtině
%%%%%%%%%%%%%%%%%%%%%%%%%%%%
	\noindent{\Large{\bfseries{Abstrakt}\\}}
	\noindent
Tato závěrečná studijní práce se zabývá návrhem a implementací jednoduché tahové skirmish hry vytvořené v herním enginu Godot. Cílem práce bylo navrhnout funkční herní systém, který zahrnuje správu herních jednotek, tahový systém, pohyb po mapě, bojový mechanismus a uživatelské rozhraní.

Praktická část práce se zaměřuje na implementaci herní logiky pomocí skriptovacího jazyka GDScript, návrh herní mapy založené na dlaždicích a práci s herními objekty. Důraz je kladen na přehlednou strukturu kódu, využití objektově orientovaného přístupu a správu herních stavů, jako je střídání hráčů a vyhodnocování soubojů.

Výsledkem práce je funkční prototyp tahové hry pro dva hráče, který demonstruje základní principy vývoje her a může sloužit jako základ pro další rozšíření, například o umělou inteligenci, nové jednotky nebo komplexnější herní mechaniky.

	
	\vspace{18pt}
	
	\noindent{\large{\bfseries{Klíčová slova}}}
	
	\noindent skirmish hra, tahová hra, Godot Engine, GDScript, herní logika, vývoj her
	
	\vspace{18pt}

%% Abstrakt v angličtině
%%%%%%%%%%%%%%%%%%%%%%%%%%%%	
	\noindent{\Large{\bfseries{Abstract}}}
	
	\noindent
This final study project focuses on the design and implementation of a simple turn-based skirmish game developed using the Godot game engine. The main goal of the project was to create a functional game system including unit management, turn handling, tile-based movement, combat mechanics, and a basic user interface.

The practical part of the project describes the implementation of game logic using the GDScript programming language, the design of a tile-based game map, and the interaction between game objects. Emphasis is placed on clear code structure, the use of object-oriented programming principles, and proper management of game states such as turn switching and combat resolution.

The result of the project is a functional prototype of a two-player turn-based game that demonstrates fundamental principles of game development and can serve as a foundation for future extensions, such as artificial intelligence, additional units, or more advanced game mechanics.

	
	\vspace{18pt}
	
	\noindent{\large{\bfseries{Keywords}}}
	
	\noindent  skirmish game, turn-based game, Godot Engine, GDScript, game logic, game development

	
	\clearpage %% Zalomení stránky

%% Stránka s generovaným obsahem
%%%%%%%%%%%%%%%%%%%%%%%%%%%%%%%%%%%%%%%	
	
	\tableofcontents %% Vygeneruje tabulku s obsahem

	\pagenumbering{arabic} %% Nastavení způsobu číslování stránek (alternativy roman | Roman)
	\setcounter{page}{1} %% Nastavení počitadla stránek


%% Stránka s úvodem
%%%%%%%%%%%%%%%%%%%%%%%%%%%%%%%%%%%%%%%		
\chapter*{Úvod}
\addcontentsline{toc}{chapter}{Úvod}

Svět stolních a deskových her mě fascinoval odjakživa. Od rychlých karetních her až po komplexní strategie trvající několik hodin, které svými mechanikami představují výzvu i pro zkušené hráče. Při výběru tématu závěrečné práce proto byla moje volba jasná – chtěl jsem propojit svou vášeň pro deskové hry s programováním a přenést zážitek z fyzického stolu do digitálního prostředí.

Rozhodování o konkrétním titulu nebylo snadné. Zvažoval jsem adaptaci sběratelských karetních her, jako je \textit{Magic: The Gathering}, kterými jsem se inspiroval již v minulosti, nebo naopak komplexních asymetrických strategií typu \textit{Root}. Nakonec jsem však zvolil "zlatou střední cestu" – hru \textit{Warhammer Warcry}.

Jedná se o tzv. skirmishovou strategii, ve které se střetávají dvě menší skupiny jednotek. Cílem je buď eliminace soupeře, nebo splnění taktických úkolů. Tato hra nabízí ideální rovnováhu: její pravidla jsou srozumitelná, ale zároveň poskytují dostatečnou hloubku pro implementaci zajímavých programátorských výzev, jako je pohyb po mřížce (gridu) nebo systém viditelnosti ve 2D prostoru.

Samotný vývoj v herním enginu Godot se ukázal být náročnější, než jsem původně předpokládal. Záhy jsem narazil na nedostatek specifických výukových materiálů a tutoriálů pro tento konkrétní typ hry, což mě donutilo hledat vlastní řešení a experimentovat. Proces byl provázen mnoha slepými uličkami a nutností opakovaně přepisovat části kódu (refactoring). Tato zkušenost však byla ve výsledku neocenitelná – naučila mě nejen lépe ovládat jazyk GDScript, ale především analyticky přemýšlet o architektuře softwarového projektu.

Předkládaná práce je rozdělena do čtyř hlavních částí. První kapitola představuje použité technologie, zejména engine Godot a grafický editor Aseprite. Druhá kapitola se věnuje návrhu architektury aplikace a struktuře herní scény. Třetí, nejobsáhlejší kapitola, detailně popisuje technickou implementaci klíčových mechanik, jako je algoritmus A* pro hledání cesty, Bresenhamův algoritmus pro viditelnost nebo systém iniciativy. Závěr práce pak shrnuje dosažené výsledky a hodnotí funkčnost vytvořeného prototypu.


%% ==========================================
%% KAPITOLA 1: POUŽITÉ TECHNOLOGIE
%% ==========================================
\chapter{Použité technologie a nástroje}

V této kapitole jsou představeny softwarové nástroje, které byly zvoleny pro realizaci projektu, a teoretické principy klíčových algoritmů.

\section{Godot Engine}
Pro vývoj hry byl zvolen herní engine Godot ve verzi 4.x. Jedná se o open-source nástroj (licence MIT), který poskytuje komplexní prostředí pro tvorbu 2D i 3D her. Oproti konkurenčním enginům, jako jsou Unity nebo Unreal Engine, vyniká Godot svou lehkostí a unikátní architekturou.

\subsection{Architektura uzlů a scén}
Základním stavebním kamenem Godotu je systém uzlů (Nodes). Každý herní objekt, ať už jde o postavu, mapu nebo zvukový přehrávač, je uzel. Uzly se skládají do stromové struktury (Scene Tree).
Tato hierarchie umožňuje vysokou modularitu – komplexní objekty (např. hráč) jsou složeny z jednodušších uzlů (Sprite, Kolizní tvar, Kamera), které lze uložit jako samostatnou scénu a instancovat ji v jiných částech projektu.

\subsection{Systém signálů}
Pro komunikaci mezi objekty využívá Godot návrhový vzor Observer, implementovaný formou signálů. To umožňuje, aby objekty vysílaly zprávy o změně svého stavu (např. \uv{jednotka dokončila pohyb}), aniž by musely znát příjemce této zprávy. Tento přístup zajišťuje oddělení logiky (decoupling) a přehlednější kód.

\section{Jazyk GDScript}
Logika hry je psána v jazyce GDScript, který byl vyvinut speciálně pro potřeby enginu Godot. Jedná se o vysokoúrovňový jazyk, jehož syntaxe je silně inspirována jazykem Python (využívá odsazování bloků kódu).

Mezi klíčové vlastnosti jazyka využité v této práci patří:
\begin{itemize}
	\item \textbf{Volitelné statické typování:} Ačkoliv je GDScript dynamický jazyk, verze 4.x klade důraz na definici typů proměnných (např. \texttt{var health: int}). To zvyšuje bezpečnost kódu, umožňuje lepší našeptávání v editoru a zvyšuje výkon aplikace.
	\item \textbf{Klíčové slovo @export:} Umožňuje vystavit proměnné ze skriptu přímo do grafického editoru (Inspektoru). Designér tak může upravovat parametry hry (síla útoku, rychlost pohybu) bez nutnosti zasahovat do zdrojového kódu.
	\item \textbf{Integrace s C++ API:} GDScript má přímý přístup k nízkoúrovňovým funkcím enginu, jako jsou matematické operace s vektory nebo fyzikální výpočty.
\end{itemize}
\section{Aseprite}
Pro tvorbu grafické stránky hry (pixel art) byl využit editor Aseprite. Jedná se o specializovaný nástroj pro tvorbu 2D spritů a animací, který je v herním průmyslu standardem pro retro grafiku.
Mezi klíčové funkce využité v tomto projektu patří:
\begin{itemize}
	\item Práce s vrstvami (Layers) pro oddělení obrysů a barev.
	\item Tvorba bezešvých textur (Seamless Tiles) pro herní mapu.
	\item Export do formátu Sprite Sheet, který Godot Engine efektivně zpracovává při vykreslování animací.
\end{itemize}

\section{Algoritmus A* (A-Star)}
Pro hledání cesty (pathfinding) v tahových strategiích je standardem algoritmus A*. Tento grafový algoritmus hledá nejkratší cestu mezi startem a cílem pomocí heuristické funkce, která odhaduje zbývající vzdálenost. V projektu je využita implementace \texttt{AStarGrid2D}, která je optimalizovaná přímo pro mřížkové mapy.

%% ==========================================
%% KAPITOLA 2: NÁVRH A ARCHITEKTURA
%% ==========================================
\chapter{Návrh a architektura aplikace}

Tato kapitola popisuje vnitřní strukturu projektu, organizaci herních objektů a vztahy mezi nimi.

\section{Struktura herní scény}
\label{sec:struktura_sceny}
Základem projektu je hlavní scéna \texttt{Game.tscn}, která funguje jako kořenový uzel a sdružuje herní logiku, mapu a uživatelské rozhraní. Organizační struktura využívá systém rodič-potomek, což je klíčové pro správné fungování transformací a relativních cest ve skriptech.

Níže uvedený seznam zobrazuje hierarchii uzlů v projektu (v závorce je uveden typ nebo připojený skript):

\begin{itemize}
	\item \textbf{Game} (Root Node)
	\begin{itemize}
		\item \textbf{TileMap} -- \textit{Vykreslování herního světa}
		\item \textbf{MovementOverlay} -- \textit{Vrstva pro zobrazení dosahu pohybu}
		\item \textbf{AttackOverlay} -- \textit{Vrstva pro zobrazení dosahu útoku}
		\item \textbf{TurnManager} (\texttt{turn\_queue.gd}) -- \textit{Správce tahů a pravidel hry}
		\item \textbf{Players} (Kontejnery pro hráče)
		\begin{itemize}
			\item player1 (\texttt{player.gd})
			\item player2 (\texttt{player.gd})
		\end{itemize}
		\item \textbf{Unit} (Kontejner pro jednotky ve scéně)
		\begin{itemize}
			\item Scout\_P1 (\texttt{scout.gd})
			\item Scout\_P2 (\texttt{scout.gd})
		\end{itemize}
		\item \textbf{CanvasLayer} (Uživatelské rozhraní)
		\begin{itemize}
			\item UnitStatPopup (\texttt{popupmenu.gd}) -- \textit{Informační okno jednotky}
			\item Initiative (\texttt{initiative.gd}) -- \textit{Panel iniciativy}
		\end{itemize}
	\end{itemize}
\end{itemize}

\section{Objektový návrh a dědičnost}
Architektura kódu je postavena na principu dědičnosti, což umožňuje snadné rozšiřování hry o nové typy jednotek bez nutnosti přepisovat základní logiku.

\subsection{Třída Unit}
Základní třída \texttt{Unit} (soubor \texttt{unit.gd}) definuje společné vlastnosti všech postav. Obsahuje:
\begin{itemize}
	\item \textbf{Atributy:} Zdraví (\texttt{health\_points}), pohyb (\texttt{movement\_points}), útok (\texttt{attack}).
	\item \textbf{Signály:} \texttt{movement\_finished}, \texttt{no\_actions\_left} pro komunikaci s manažerem hry.
\end{itemize}

\subsection{Specializace - Třída Scout}
Konkrétní jednotky, jako je Průzkumník (soubor \texttt{scout.gd}), dědí ze třídy \texttt{Unit}. V metodě \texttt{\_ready()} přepisují základní statistiky:

\begin{lstlisting}[style=Python, caption={Ukázka dědičnosti ve třídě Scout}]
extends Unit
class_name Scout

func _ready():
	# Nastavení statistik specifických pro Scouta
	movement_points = 6
	health_points = 8
	toughness = 3
	type = "Scout"
\end{lstlisting}

%% ==========================================
%% KAPITOLA 3: IMPLEMENTACE
%% ==========================================
\chapter{Implementace herních mechanik}

Tato kapitola detailně popisuje programová řešení klíčových prvků hry. Zaměřuje se na algoritmy pro správu tahů, výpočet pohybu v mřížce a matematický model soubojového systému včetně řešení viditelnosti (Line of Sight).

\section{Řízení herní smyčky (Turn System)}
Jádrem herní logiky je skript \texttt{turn\_queue.gd}, který funguje jako centrální stavový automat. Hra se dělí na kola, která se skládají z tahů jednotlivých hráčů.

\subsection{Iniciativa}
Pořadí hráčů není fixní, ale určuje se na začátku každého kola pomocí systému iniciativy. Každý hráč disponuje sadou virtuálních kostek (reprezentovaných polem \texttt{dice\_pool} ve třídě \texttt{Player}).

Metoda \texttt{roll\_initiative\_dice()} vygeneruje pro každou kostku náhodnou hodnotu v rozsahu 1–6. Hráč s nejnižším počtem stejných hodnot získává právo prvního tahu. Tento prvek náhody nutí hráče adaptovat strategii na každé nové kolo.

\section{Pohyb a Pathfinding}
Pohyb jednotek po herní ploše je realizován pomocí grafového algoritmu A* (A-Star), který hledá nejkratší cestu mezi startovní a cílovou dlaždicí s ohledem na překážky.
\clearpage
\subsection{Konfigurace navigační mřížky}
V projektu je využit uzel \texttt{AStarGrid2D}, který je optimalizován pro mřížkové mapy. Při inicializaci (\texttt{\_ready}) se graf synchronizuje s vrstvou \texttt{TileMap}:

\begin{lstlisting}[style=Python, caption={Inicializace A* mřížky v turn\_queue.gd}]
astar_grid = AStarGrid2D.new()
astar_grid.region = tile_map.get_used_rect()
astar_grid.cell_size = Vector2(16, 16)
# Zákaz diagonálního pohybu pro zachování čtvercové metriky
astar_grid.diagonal_mode = AStarGrid2D.DIAGONAL_MODE_NEVER
astar_grid.update()
\end{lstlisting}

Byla zvolena **Manhattanská metrika** (pohyb pouze ve směru os X a Y), což zjednodušuje počítání vzdálenosti pro hráče – jeden krok vždy stojí jeden akční bod.

\subsection{Interpolace pohybu (Tweening)}
Samotná změna pozice jednotky není ve hře okamžitá. Aby byl pohyb vizuálně plynulý, využívá se objekt \texttt{Tween}, který provádí lineární interpolaci souřadnic v čase.

Metoda \texttt{move\_along\_path} ve třídě \texttt{Unit} přijímá pole bodů cesty a postupně mezi nimi animuje sprite jednotky:

\begin{lstlisting}[style=Python, caption={Plynulý pohyb pomocí Tween (unit.gd)}]
func move_along_path(path: Array[Vector2]) -> void:
	if path.is_empty(): return
	
	actions -= 1
	var tween = create_tween()
	
	for point in path:
		# Animace přesunu na další bod cesty trvá 0.25 sekundy
		tween.tween_property(self, "global_position", point, 0.25)\
			.set_trans(Tween.TRANS_SINE)\
			.set_ease(Tween.EASE_IN_OUT)
	
	# Callback po dokončení pohybu
	tween.tween_callback(Callable(self, "_on_move_finished"))
\end{lstlisting}

\section{Bojový systém}
Boj je interakcí mezi útočníkem a obráncem, která podléhá pravidlům o dosahu, viditelnosti a statistikách jednotek. 

\subsection{Algoritmus vyhodnocení útoku}
Samotný průběh boje je determinován porovnáním statistik útočníka a obránce. Systém nejprve určí prahovou hodnotu pro úspěšný zásah (Target Number) na základě vztahu mezi silou útočníka (\texttt{Strength}) a odolností obránce (\texttt{Toughness}).

Platí následující pravidla pro určení hodnoty potřebné k zásahu:
\begin{itemize}
	\item Pokud $\texttt{Strength} > \texttt{Toughness}$: Útočník má výhodu, pro zásah stačí hodnota \textbf{3+}.
	\item Pokud $\texttt{Strength} = \texttt{Toughness}$: Síly jsou vyrovnané, standardní hodnota pro zásah je \textbf{4+}.
	\item Pokud $\texttt{Strength} < \texttt{Toughness}$: Obránce má převahu, útočník musí hodit \textbf{5+}.
\end{itemize}

Následně proběhne simulace hodu $N$ šestistěnnými kostkami, kde $N$ odpovídá atributu \texttt{Attack} útočníka. Každá kostka se vyhodnocuje samostatně:

\begin{enumerate}
	\item \textbf{Neúspěch:} Hodnota kostky je nižší než stanovená prahová hodnota. Žádné poškození se neaplikuje.
	\item \textbf{Zásah (Hit):} Hodnota je rovna nebo vyšší než práh, ale nižší než 6. Je uděleno poškození ve výši atributu \texttt{Hit}.
	\item \textbf{Kritický zásah (Crit):} Na kostce padla hodnota 6. Místo standardního poškození se aplikuje hodnota \texttt{Crit}, která reprezentuje silnější úder.
\end{enumerate}

Celkové poškození je součtem výsledků všech úspěšných hodů.

\subsection{Atributy jednotek}
Každá jednotka je definována sadou parametrů, které ovlivňují výsledek boje. Tyto atributy jsou definovány ve třídě \texttt{Unit} a upravovány v potomcích (např. \texttt{Scout}).

\begin{itemize}
	\item \textbf{Health Points (HP):} Představují celkovou vitalitu jednotky. Při snížení hodnoty na 0 je jednotka vyhodnocena jako zničená a následně odstraněna z herní scény.
	\item \textbf{Attack:} Kvantifikuje objem útoku. Číselná hodnota odpovídá počtu kostek, kterými se hází při vyhodnocení souboje.
	\item \textbf{Far:} Definuje maximální efektivní dostřel jednotky (vyjádřeno v počtu políček mřížky).
	\item \textbf{Strength (Síla) \& Toughness (Odolnost):} Dvojice protichůdných atributů (ofenzivní vs. defenzivní). Jejich vzájemným porovnáním se určuje obtížnost hodu na zásah.
	\item \textbf{Hit \& Crit:} Konstanty definující výši způsobeného poškození. \textit{Hit} se aplikuje při standardním zásahu, zatímco \textit{Crit} se aplikuje při padnutí hodnoty 6 na hrací kostce.
\end{itemize}

\subsection{Detekce viditelnosti (Line of Sight)}
Aby mohla jednotka zaútočit na dálku, musí mít na cíl přímou viditelnost. Protože hra nefunguje na fyzikálním enginu, ale na mřížce, nemohl být použit standardní \texttt{RayCast2D}.

Místo toho byla implementována vlastní variace **Bresenhamova algoritmu**. Tento algoritmus matematicky simuluje přímku mezi dvěma body v mřížce a kontroluje, zda některá z buněk na této přímce neobsahuje překážku (dle vlastnosti \textit{Custom Data: Walkable} v TileSetu).

Metoda \texttt{has\_line\_of\_sight\_tile} prochází buňky od útočníka k cíli. Pokud narazí na zeď, vrátí \texttt{false} a útok je znemožněn.
\clearpage

\begin{lstlisting}[style=Python, caption={Algoritmus kontroly viditelnosti (zjednodušeno)}]
func has_line_of_sight_tile(start: Vector2i, end: Vector2i) -> bool:
	var dx = abs(end.x - start.x)
	var dy = -abs(end.y - start.y)
	var err = dx + dy
	
	while true:
		# Pokud je na aktuální dlaždici překážka, není vidět
		var data = tile_map.get_cell_tile_data(start)
		if data and not data.get_custom_data("Walkable"):
			return false
			
		if start == end: break
		
		# Výpočet dalšího kroku v mřížce
		var e2 = 2 * err
		if e2 >= dy:
			err += dy
			start.x += sx
		if e2 <= dx:
			err += dx
			start.y += sy
	return true
\end{lstlisting}

\section{Vizualizace a uživatelské rozhraní (UI)}
Komunikace hry s hráčem probíhá ve dvou rovinách: přímo v herním prostoru (zvýraznění možností pohybu) a v statické vrstvě rozhraní (HUD).

\subsection{Interaktivní vrstvy (Overlays)}
Pro vizualizaci taktických možností, jako je dosah pohybu nebo útoku, hra nevyužívá drahé instancování objektů pro každé políčko. Místo toho jsou použity dedikované vrstvy typu \texttt{TileMapLayer}:
\begin{itemize}
	\item \textbf{MovementOverlay:} Zobrazuje modře políčka, na která může jednotka dojít.
	\item \textbf{AttackOverlay:} Zobrazuje červeně políčka, která jsou v dosahu útoku.
\end{itemize}

Tyto vrstvy jsou ve stromu scény umístěny nad základní mapou, ale pod jednotkami. Při výběru jednotky skript \texttt{turn\_queue.gd} vypočítá validní souřadnice (pomocí A* pro pohyb nebo Manhattanské vzdálenosti pro útok) a na příslušné souřadnice v overlay vrstvě umístí dlaždici s poloprůhlednou texturou:

\begin{lstlisting}[style=Python, caption={Vykreslení overlaye pohybu}]
func draw_movement_overlay(unit, reachable_points):
	overlay_map.clear() # Smazání předchozího stavu
	for point in reachable_points:
		# Nastavení modré dlaždice na souřadnici
		overlay_map.set_cell(point, SOURCE_ID, ATLAS_COORDS)
\end{lstlisting}

Toto řešení je vysoce výkonné, protože \texttt{TileMapLayer} vykresluje všechny dlaždice v jednom draw-callu (vykreslovacím cyklu).

\subsection{Statické rozhraní (HUD)}
Informační prvky, které se nemají pohybovat s kamerou (např. statistiky), jsou umístěny ve vrstvě \texttt{CanvasLayer}.

Klíčovým prvkem je skript \texttt{popupmenu.gd}, který reaguje na signál \texttt{unit\_selected}. Po kliknutí na jednotku se dynamicky načtou její aktuální statistiky do připravených textových polí (\texttt{Label}) a zobrazí se informační okno.	
	

%% ZÁVĚR PRÁCE
%%%%%%%%%%%%%%%%%%%%%%%%%%%%%%%%%%%%%%%
\chapter*{Závěr}
\addcontentsline{toc}{chapter}{Závěr}

Cílem této závěrečné práce bylo navrhnout a naimplementovat tahovou strategickou hru v herním enginu Godot. Tento cíl byl splněn. Výsledná aplikace obsahuje funkční systém střídání tahů, pohyb jednotek po mřížce pomocí algoritmu A* a komplexní soubojový systém založený na statistikách jednotek.

Během vývoje jsem si prohloubil znalosti objektově orientovaného programování v jazyce GDScript a naučil se řešit specifické problémy herního vývoje, jako je oddělení herní logiky od vizualizace nebo práce s diskrétní geometrií na mřížce.

Vytvořený projekt představuje solidní základ, na kterém lze v budoucnu stavět, ať už přidáním umělé inteligence, nebo rozšířením herního obsahu. Práce prokázala, že Godot Engine je mocným nástrojem pro tvorbu 2D her, který umožňuje efektivní vývoj i pro jednotlivce.

%% LITERATURA
%%%%%%%%%%%%%%%%%%%%%%%%%%%%%%%%%%%%%%%
\begin{thebibliography}{99}
	\bibitem{godot_docs} \textit{Godot Engine 4.x Documentation} [online]. Godot Engine Project, 2024 [cit. 2026-01-05]. Dostupné z: \url{https://docs.godotengine.org}
	
	\bibitem{gdscript_ref} \textit{GDScript reference} [online]. Godot Engine Project, 2024 [cit. 2026-01-05]. Dostupné z: \url{https://docs.godotengine.org/en/stable/tutorials/scripting/gdscript/gdscript_basics.html}
	
	\bibitem{astar_algo} \textit{A* Search Algorithm} [online]. GeeksforGeeks, 2024 [cit. 2026-01-05]. Dostupné z: \url{https://www.geeksforgeeks.org/a-search-algorithm/}
    
    \bibitem{bresenham} \textit{Bresenham's line algorithm} [online]. Wikipedia, 2024 [cit. 2026-01-05]. Dostupné z: \url{https://en.wikipedia.org/wiki/Bresenham%27s_line_algorithm}
\end{thebibliography}

%% SEZNAMY OBRÁZKŮ A TABULEK
%%%%%%%%%%%%%%%%%%%%%%%%%%%%%%%%%%%%%%%
\begingroup
    \let\clearpage\relax
    \listoffigures
    \vspace{2cm}
    \listoftables
\endgroup

\end{document}